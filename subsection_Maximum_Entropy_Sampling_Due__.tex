\subsection{Maximum Entropy Sampling}

Due to the size of the datasets in this study, at each sampling stage we draw multiple molecules - 200 for the malaria and one-dose datasets which both contain around 20,000 molecules - and 1000 for the CEP dataset, which contains 2 million molecules. Additionally for the CEP dataset, in order to penalize adding sets of molecules which are very similar (and thus contribute less to the entropy) we remove redundent molecules by applying a clustering based upon the Tanimoto distance, with molecules being clustered with a tolerance of 0.9 distance and only one representative of each cluster added to the training set.  The effect of removing redundant molecules is particularly strong in very large datasets such as the CEP, and whilst we tested this method on the other data sets, we found that it had little or no effect upon the outcome, so for these datasets we report the results without the clustering.


For each of the results shown in Figure \ref{fig:max_entropy}, it can be seen that actively collecting molecules using the maximum entropy sampling methodology affords an improvement in the speed of discovery of maximally informative sets of molecules, with the RMSE curve both dropping much faster to a low error, compared to the Monte Carlo sampling.  For the largest of these sets, the CEP Data Set, the trajectory of this sampling is much noisier for the Monte Carlo sampler, than the maximum entropy sampler.  The smoothness of the maximum entropy sampler is important when considering the application of these methods in `live' situations; sampling 

This result is not surprising when the distribution of pairwise distances derived using the network based components is considered. The result is most pronounced for the Malaria dataset and the One-Dose dataset, which show the most skewed distribution.  The One-Dose dataset has the lowest mean of the three sets, which suggests that molecules picked randomly will most likely have a small 'component distance', this suggests that they will be similar in the eyes of the neural network, and thus contribute little information entropy to the training set. Whilst the Malaria dataset has a higher mean, it exhibits a long-tailed distribution, which is again problematic for a Monte Carlo sampling approach.  In comparison to those, pairwise distances within the Clean Energy Project dataset are fairly close to being normally distributed.  This is realized in the results of the sampling, where Monte Carlo - despite displaying significantly more variance in its performance - does not perform as poorly when compared to the maximum entropy sampling methodology.

It is also worthwhile recalling that this method selects molecules without considering their contribution to the overall error of the training set; instead utilizing the predictive uncertainty of the model for selecting which molecules to add to the training set. Thus, it represents a purely information-based approach and clearly demonstrates the power of exploiting the underlying information landscape for intelligently constructing informative libraries.