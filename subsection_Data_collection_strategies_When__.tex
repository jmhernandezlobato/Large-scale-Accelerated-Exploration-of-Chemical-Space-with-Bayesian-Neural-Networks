\subsection{Data collection strategies}

When exploring chemical space, there are two main tasks of interest. The first one consists in finding quickly molecules that achieve high values of a particular fitness function. In some cases the goal may be to find just a single optimal molecule, but in many ---especially in high-throughput screening scenarios \cite{Pyzer_Knapp_2015a}--- the desired outcome is to find 
an enriched set of molecules that achieve the highest possible fitness values. For solving this type of problems, we propose to use a data collection strategy called Thompson sampling \cite{Thompson_1933}. This is a simple heuristic that automatically balances exploration and exploitation \cite{Chapelle2011} and that is highly scalable and easy to implement in a distributed setting.

The first one is to find data that will lead to low prediction error as quickly as possible. This is particularly helpful when we need accurate predictions over a wide range of possible inputs. One important case within this category is the prediction of toxcity. 

Here, the accuracy of a model is important and  also not a classic optimization problem. For this first task we propose to use the maximum entropy sampling heuristic, which can be shown to select the most informative molecules for solving
the prediction problem.  The second task is a classic optimization challenge --- we want to find molecules that achieve high values for some fitness
function.  In some cases this may be one molecule, but in many --- especially in high-throughput screening scenarios \cite{pyzer-knapp_what_2015} --- the
desired outcome is to find the most enriched set of molecules. For this task, we propose to use Thompson sampling \cite{thompson_likelihood_1933}, a simple
heuristic that achieves a strong balance of exploration and exploitation \cite{Chapelle2011}.