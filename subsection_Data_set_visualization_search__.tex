\subsection{Data set visualization: search landscape plots}

In this section we present a 2D visualization of the search space for each of
the previous data sets. We refer to the resulting plots as \emph{search
landscapes}. These plots are useful to visually understand the individual
characteristics of each search problem. 

We use an
auto-encoder neural network to perform a non-linear dimensionality reduction of the molecule feature vectors
into a 2D space \cite{Hinton_2006}. The auto-encoder network has a mirrored
funneling architecture formed by an encoder and decoder network connected to
each other, see the plot in the
right of Figure \ref{fig:stacked_rbms}. The bottom layer has the same dimension as the data. Subsequent
layers reduce the dimension up to two. The last group of layers
increase the dimension back to the original one. 
The data is propagated forward from the bottom to the top layer. The network
weights are adjusted by backpropagation so that the reconstruction at the top
layer is as close as possible to the original data. After
the backpropagation process, the bottom part of the mirrored funnel architecture, that is, the
encoder, is used to perform a non-linear projection of the molecules into a 2D
space. All the activation functions in the auto-encoder are logistic functions. Because of this, the data is projected into the $[0,1]\times[0,1]$ space. Before running the backpropagation process, bidirectional
connections between pairs of consecutive layers (see the left plot in Figure
\ref{fig:stacked_rbms}) are sequentially pre-trained as
generative models for the data in a layer-wise manner \cite{Hinton_2006}. This pre-training step
produces a sensible initialization of the network weights before the fine-tuning by backpropagation.

The 2D projection of the molecules is used to generate the search
landscape plots. In these plots chemical space is described by the coordinates of the 2D projection,
which are bounded between 0 and 1. This space is split into 25 square areas of size $0.2 \times 0.2$ whose 
color is determined by the mean value of the target variables associated with the molecules in the area. We also visualize the empirical distribution of molecules by using a kernel density estimate whose contour lines are depicted in the plots. Additionally, the top 100 molecules, ranked
according to their target variables, are shown as blue points in the plots.  

By
analyzing the distribution of the data, alongside the location of extreme
molecules, it is easy to classify the data into types of information landscape.
In Figure \ref{fig:info_landscapes}, we can see the information landscapes
created to describe each of the data sets used within this study.  The Clean
Energy Project dataset is striking in the strength of the input/property
relationship --- the majority of the top molecules are concentrated in a
relatively small area of chemical space.  Additionally, in this dataset, there
is a clear trend for worsening performance as the Y-axis is ascended.  This is
not the case for the other two datasets, in which the `good' areas of chemical
space are more randomly distributed.  It is interesting to observe the One-Dose
data set, in which there are two clear families of extreme molecules, one
towards the top of the plot and one towards the bottom left.  This could be
indicative of two different approaches taken in the literature to approach the
problem of developing good treatments for this particular type of cancer.

