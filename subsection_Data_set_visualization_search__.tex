\subsection{Data set visualization: search landscape plots}

In this section we present a 2D visualization of the search space for each of
the previous data sets. We refer to the resulting 2D plots as \emph{search
landscapes}. These plots are useful to visually understand the individual
characteristics of each search problem. The plots are generated by using an
auto-encoder neural network that performs a non-linear projection of the
molecules into a 2D space \cite{Hinton_2006}. The network has a mirrored
funneling architecture in which the bottom layer has the same dimension as the
data, subsequent layers reduce the dimension up to two and then, the last group
of layers increase the dimension back the one of the original data, see the
plot in the right of Figure \ref{fig:stacked_rbms}.  All the activation
function in the network are logistic functions, which produce values in the
$[0,1]$ interval.

The data is propagated forward from the bottom to the top layer and the network
weights are adjusted by backpropagation so that the reconstruction at the top
layer is as close as possible to the original data at the bottom layer. After
this process, the bottom part of the mirrored funel architecture is used to
perform a non-linear projection of the molecules into a 2D space.

Before performing the
backpropagation procedure, bidirectional connections between pairs of
consecutive layers (see the left plot in Figure \ref{fig:stacked_rbms}) are
sequentially pre-trained in a layer-wise manner as generative models for the
data \cite{Hinton_2006}. This pre-training process produces better
initializations of the network weights than other approaches.

The 2D projection of the molecules are then used to generate the search
landscape plots.

The underlying information derived from these network derived components can be
further investigated through the use of a plot we call an 'information
landscape'.  In this plot, chemical space (as described by the components,
which are bound between 0 and 1) is split into areas of size 0.2x0.2 arbitrary
units. The mean value for the targets contained within each area is calculated,
and determines the color of that area.  In order to visualize the distribution
of molecules within this depiction of chemical space, a kernel density estimate
is used to plot contour lines.  Additionally, the top 100 molecules, ranked
according to their target value, are plotted as points on the surface.  By
analyzing the distribution of the data, alongside the location of extreme
molecules, it is easy to classify the data into types of information landscape.
In Figure \ref{fig:info_landscapes}, we can see the information landscapes
created to describe each of the data sets used within this study.  The Clean
Energy Project dataset is striking in the strength of the input/property
relationship --- the majority of the top molecules are concentrated in a
relatively small area of chemical space.  Additionally, in this dataset, there
is a clear trend for worsening performance as the Y-axis is ascended.  This is
not the case for the other two datasets, in which the `good' areas of chemical
space are more randomly distributed.  It is interesting to observe the One-Dose
data set, in which there are two clear families of extreme molecules, one
towards the top of the plot and one towards the bottom left.  This could be
indicative of two different approaches taken in the literature to approach the
problem of developing good treatments for this particular type of cancer.

