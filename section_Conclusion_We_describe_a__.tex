\section{Conclusion}
We describe a method which utilizes techniques derived from information theory and machine learning to guide an intelligent search of local areas of chemical space, and its application to the discovery of novel photovoltaic materials and therapeutic molecules.  In order to understand the performance of these intelligent searching methods, we also present a novel component reduction algorithm, based upon the use of unsupervised neural networks.  This demonstrates the diversity of the datasets through the eyes of a neural network, and provides a means of analysing diversity through the components which are most strongly expressed through the neural network.  

Our results demonstrate how the underlying uncertainty in predictive values can be exploited through the use of maximum entropy sampling to provide the which optimially balance predictive power against the size of the molecular library. If the search can be formulated as a classic optimization problem, we show how Thompson sampling can be used to quickly and robustly locate extreme molecules by balancing exploration of the underlying information, with exploitation of the predictive model provided by a Bayesian neural network. This resulted in significant increases in the rate of discovery, displaying significant potential for improving the current model of molecular discovery, especially within the virtual realm.
