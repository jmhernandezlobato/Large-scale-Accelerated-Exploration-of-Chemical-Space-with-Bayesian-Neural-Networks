\section{Methods}

Adaptive design tools include three main components. First, a
\emph{probabilistic model} that describes the fitness values obtained by
performing experiments. Second, an \emph{inference algorithm} that combines the
probabilistic model with the data collected so far to make predictions on new
data and finally, a \emph{data collection strategy} that is used to make
intelligent decisions about what experiments to perform next. We choose our
probabilistic models to be Bayesian neural networks, which are highly scalable
and often make very accurate predictions with large amounts of data. For
inference we use the algorithm probabilistic back-propagation, which is a very
recent technique that has low computational cost and produces accurate
estimates of uncertainty in the predictions of neural networks.
In the following sections we describe these individual components.

%Random searches have been used for the exploration of chemical spaces \cite{23548177},
%and include results for a random search as a benchmark for these methods, since
%for a method to approach `intelligence', it surely must give significant
%improvements in efficiency over a random (Monte Carlo) approach.

%Our adaptive design methods balance
%5exploration of a wide range of molecules (often thought of in terms of
%diversity maximization
%\cite{Blaney_1997,Wang_2009,Fitzgerald_2006,reker_activelearning_2015})
%with explotation of the most promissig molecules.


