\subsubsection{Thompson sampling}

Thompson sampling (TS) \cite{Thompson_1933} is a simple heuristic for efficiently identifying the optimal element in a set. Solving this problem requires to find an optimal exploration/explotation trade-off. A strong property of TS  is that this is performed automatically without the need for the introduction of additional parameters, which would themselves require optimization.
We achieve this with the following algorithm:
\begin{enumerate}
\item A very small number of molecules are selected randomly, and ground truth values are calculated
\item A Bayesian neural network is trained with PBP on this training data
\item The model parameters are sampled from their posterior distribution, providing a set of deterministic neural networks
\item Each deterministic neural network makes predictions for all the remaining molecules.
\item For each deterministic neural network, the set of unique molecules with highest score is selected,  ground truth values calculated, and added to the training set.
\item Steps 2-5 are repeated
\end{enumerate}
By sampling the weight distributions in the Bayesian neural network, we produce a set of deterministic neural networks, the weights of which vary across the set in a manner directly related to the uncertainty of the probabilistic model in their value.  Thus, a balance in the search for extreme molecules is achieved without directly imposing any external measure of molecular diversity - which is in itself an area of intense study \cite{Maldonado_2006}.