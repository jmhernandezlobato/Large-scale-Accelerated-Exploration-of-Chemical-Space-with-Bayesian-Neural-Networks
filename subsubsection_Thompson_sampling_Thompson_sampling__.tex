\subsubsection{Thompson sampling}

Thompson sampling (TS) \cite{Thompson_1933} is a simple heuristic for identifying the optimal element in a set. Solving this problem efficiently, that is, with a small number of evaluations of the fitness function, requires to attain an optimal exploration/exploitation trade-off. TS does this automatically, without having to use additional parameters for specifying the trade-off explicitly, which would themselves require optimization. Another advantage of Thompson sampling is that it can collect batches of data in parallel. Furthermore, TS can be implemented so that the operations performed for the selection of each element in the batch can be done independently in different nodes in a computer cluster. This is important to speed up computations, especially when the batch size is very large and the library of candidate molecules contains millions of elements. The following lines describe the implementation of Thompson sampling for the efficient collection of $N$ new measurements in parallel:
\begin{enumerate}
\item A small number of molecules are selected uniformly at random from the library of candidate molecules and their ground truth values are calculated.
\item A Bayesian neural network is trained with PBP on all the data collected so far.
\item The network weights are sampled $N$ times from the approximation to their posterior distribution. This results in $N$ deterministic neural networks with known weight values.
\item Each of the $N$ deterministic neural networks makes predictions on all the remaining molecules.
\item For each of the $N$ deterministic neural networks, we select the set of top $N$ molecules with highest predicted scores.
\item We select a unique set of $N$ molecules with highest rankings among the $N$ sets of size $N$ generated by each deterministic neural network.
\item Steps 2-5 are repeated
\end{enumerate}
By sampling the weight distributions in the Bayesian neural network, we produce a set of deterministic neural networks, the weights of which vary across the set in a manner directly related to the uncertainty of the probabilistic model in their value.  Thus, a balance in the search for extreme molecules is achieved without directly imposing any external measure of molecular diversity - which is in itself an area of intense study \cite{Maldonado_2006}.