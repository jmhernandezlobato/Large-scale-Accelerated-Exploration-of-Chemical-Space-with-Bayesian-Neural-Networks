\subsection{Thompson Sampling}

In this section we evaluate the gains produced by Thompson sampling (TS) in a simulated high throughput virtual screening setting. For this, we sequentially sample molecules from a library of candidate molecules. After each sampling step, we calculate the 1\% recall, that is, the fraction of the top 1\% of molecules from the original library that are found among the sampled ones. Each sampling step consists in selecting a batch of molecules among those that have not been sampled so far. In the Malaria and One-dose data sets we use batches of size 200. These data sets contain each one about 20,000 molecules. The CEP data set contains by contrast 2 million molecules. In this latter case, we use batches of size 1000. 

We first compare the performance of TS with that of two simple baselines. The first one, \emph{greedy}, is a sampling strategy that only considers exploitation and does not perform any exploration. We implement this approach by selecting molecules according to the average of the probabilistic predictions generated by PBP. That is, the greedy approach ignores any variance in the predictions of PBP and generates batches by just ranking molecules according to the mean of the Gaussian predictive distribution given by PBP. The second baseline is a Monte Carlo approach in which the batches of molecules are selected uniformly at random. These two baselines are examples of techniques that, as the proposed TS method, can be easily applied in a large scale setting in which the library of candidate molecules contains millions of elements.

In the Malaria and One-dose data sets we average across 50 different realizations of the experiments. By contrast, the CEP data set is 100 times larger and in this case we report results for a single realization of the experiment (in a second realization we obtained similar results). 
Figure \ref{fig:thompson_1pc} shows the results obtained by each method in the data sets from 
Section \ref{sec:data_sets}. TS significantly outperforms the Monte Carlo approach, and also offers increased performance and robustness than the greedy sampling methodology. This shows the importance of building in exploration into the sampling strategy, rather than relying on purely exploitative methods. The greedy approach performs best in the CEP data set. In this case, greedy initially finds better molecules than TS. However, after a while TS overtakes greedy. The reason for this is likely to be that an initially discovered area of chemical space with promising molecules starts to become exhausted.

It is helpful in cases such as this to think of the early difference in performance between the greedy and the Thompson strategies as a cost for building in exploration, which is eventually paid back when the 'prime' area of chemical space initially discovered by the greedy algorithm is exhausted.  It is, of course, possible to construct an example where all the molecules of interest are in one area of chemical space, and thus any time spent exploring other areas of chemical space is wasted.  In this regard, we point out that it is virtually impossible to construct an accurate information landscape without having all the information already collected, and that even with strongly localized libraries - such as the Clean Energy Project - a greedy strategy is quickly surpassed by Thompson sampling. 
  

This effect can be seen more clearly in Figure \ref{fig:cep_lib}.  The greedy strategy (green) initially performs exceptionally well, with a significant enrichment over both Monte Carlo and Thompson sampling strategies.  This advantage quickly decays, however, as the Thompson sampling benefits from the exploration performed, and the derived improved description of potential desirable areas of chemical space. We emphasize that even the Clean Energy Project, with its strongly localized cluster of desirable molecules, demonstrates deficiencies in the greedy sampling strategy.  Indeed, some of the strength shown by the greedy sampling strategy can be attributed to the sheer size of the dataset, with respect to the others investigated within this study.  It is logical that a method whose strength lies in exploiting localized knowledge of an area of chemical space in which desirable molecules have been found will benefit from a dataset which contains a large number of molecules, and which has been designed combinatorially. It should also be noted that the success of a greedy methodology is also dependent on the existence of a wide-necked funnel in the information landscape - if a cluster of desirable molecules in chemical space is too tight, it is unlikely to be sampled in the initialization of the method.  Thus, it is unlikely to be discovered - at least in an `intelligent' manner - by a greedy sampling methodology; a situation which is strongly undesirable when building libraries for which each point is expensive to discover the ground truth value.
  
In order to put the importance of these results into context, we consider the saving in calculation time for the Clean Energy Project dataset. Thompson sampling, displays $>30x$ faster discovery than the Monte Carlo search, which is in itself faster than the exhaustive enumeration used in the initial exploration of this dataset. Even assuming a discovery rate of 30 times as fast as the initial Clean Energy Project, which we belive to be conservative, 34,000 CPU years would have been saved in exploring this part of chemical space.  

Both the One-Dose and Malaria datasets contain around 20,000 molecules; yet by using Thompson sampling, we can locate 70\% of the highly active molecules in both sets, by sampling only 600 molecules. This represents a huge reduction in the discovery time for new theraputic molecules, not to mention a significant saving in the economic costs associated with synthesiszing and testing these molecules.
