
Chemical space is so large as to make brute force searching for molecules with
improved properties infeasible. Adaptive design can accelerate the discovery
process by sequentially identifying the most useful experiments to be performed
next. However, existing adaptive design methods have shortcomings that limit
their applicability to the molecule search problem. First, they lack
scalability and cannot work with the large amounts of data that are required to
successfully navigate chemical space. Second, they are unable to learn feature
representations for the data, which reduces their statistical efficiency. To
avoid these limitations, we propose an alternative approach based on the
combination of Bayesian neural networks with probabilistic active learning. Our
methods are computationally and statistically efficient by leveraging on recent
advances in approximate Bayesian inference. Thompson
sampling is used to quickly identify molecules with optimal properties.
Maximum entropy sampling is used
to find small sets of molecules with good interpolation and extrapolation properties.
Our methods can be implemented in a parallel and distributed way. This enables us to handle
massive libraries of candidate molecules and to collect data efficiently using very large batch sizes.
Our techniques generate enriched libraries of compounds in a fraction
of the time required by a stochastic (Monte Carlo) search approach and with
more robustness than a purely greedy search strategy.